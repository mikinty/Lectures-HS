The purpose of this document is to provide practice for doing basic recursion calculations.

\subsection{Introduction}
We are interested in calculating values for a recurrence relation described by:
\begin{equation}
    a_n = c_{n-1}a_{n-1}+c_{n-2}a_{n-2}+\cdots+c_0a_0
\end{equation}
where $a_i$ is a term in the sequence $\{a_i\}_{i=1}^{N}$, and $c_i$ is some function (it's usually a constant for our purposes).

In order to be able to build up our sequence $\{a_i\}_{i=1}^{N}$, all we need to do is have the necessary \textbf{initial conditions}, and then use the \textbf{recurrence relation} to figure out more terms. 

\subsubsection{Initial Conditions}
What I mean by initial condition is the starting values you use for your sequence. If you don't have the necessary information to compute the next term, then how will you be able to find more terms? For example, let's say our recurrence relation was

\begin{equation}
    a_n = 3a_{n-2}+4a_{n-2}, \quad n\geq2, a_0 = 2
\end{equation}

We can try to figure out $a_2$, but what we quickly realize is 

$$a_2 = 3a_1+4a_0 = 3(???)+8$$

...we are stuck and we can't continue. This is what I mean by you need sufficient initial conditions in order to figure out the entire sequence. 

Another example that gives you partial information about the sequence rather than the entire sequence is when there is some sort of parity involved. Consider this example:

\begin{equation}
    a_n = 2a_{n-2}, \quad n\geq 2, a_0=3
\end{equation}
We can easily compute that $a_2=6, a_4=12, a_6=24, ...$, but we don't know anything about the odd terms! Therefore, even though we are able to get an infinite sequence from our initial conditions, it isn't \textit{complete} enough so that we can figure out the entire sequence. These type of sequences appear in many applications, including solving for coefficients in solutions of differential equations.

\clearpage

\subsection{Problems}
Compute the first couple terms of the following recurrence relations (fill out the table)

\begin{problem}
\begin{equation}
    a_n = a_{n-1}+a_{n-2}, \quad n\geq2, a_0=1, a_1=1
\end{equation}
\begin{table}[H]
    \renewcommand{\arraystretch}{1.5}
    \centering
    \begin{tabular}{|c|D{1in}|}
        \hline
        $n$ & $a_n$ \\ \hline
        2 &  \\ \hline
        3 & \\ \hline
        4 & \\ \hline
        5 & \\ \hline
        6 & \\ \hline
        7 & \\ \hline
        8 & \\ \hline
    \end{tabular}
\end{table}

Does this sequence look familiar?
\end{problem}

\begin{problem}
\begin{equation}
    a_n = a_{n-1}+a_{n-2}, \quad n\geq2, a_0=2, a_1=4
\end{equation}
\begin{table}[H]
\renewcommand{\arraystretch}{1.5}
    \centering
    \begin{tabular}{|c|D{1in}|}
        \hline
        $n$ & $a_n$ \\ \hline
        2 &  \\ \hline
        3 & \\ \hline
        4 & \\ \hline
        5 & \\ \hline
        6 & \\ \hline
        7 & \\ \hline
        8 & \\ \hline
    \end{tabular}
\end{table}

Why doesn't this look like Fibonacci even though it's defined the exact same way?
\end{problem}

\clearpage

\begin{problem}
\begin{equation}
    a_n = 2a_{n-1}, \quad n\geq2, a_0=1
\end{equation}
\begin{table}[H]
\renewcommand{\arraystretch}{1.5}
    \centering
    \begin{tabular}{|c|D{1in}|}
        \hline
        $n$ & $a_n$ \\ \hline
        2 &  \\ \hline
        3 & \\ \hline
        4 & \\ \hline
        5 & \\ \hline
        6 & \\ \hline
        7 & \\ \hline
        8 & \\ \hline
    \end{tabular}
\end{table}

Can you guess the explicit formula for $a_n$? (so for example $a_n = 3n$ is explicit and not recursive (not dependent on previous terms))
\end{problem}

\begin{problem}
\begin{equation}
    a_n = 2a_{n-1}-4a_{n-2}+9a_{n-3}, \quad n\geq3, a_0=-1, a_1=2, a_2=-3
\end{equation}
\begin{table}[H]
\renewcommand{\arraystretch}{1.5}
    \centering
    \begin{tabular}{|c|D{1in}|}
        \hline
        $n$ & $a_n$ \\ \hline
        3 & \\ \hline
        4 & \\ \hline
        5 & \\ \hline
        6 & \\ \hline
        7 & \\ \hline
    \end{tabular}
\end{table}
\end{problem}

If you want, you can figure out a \textit{linear recurrence} for most of these problems. The process for figuring that out is explained in the recursion lecture.