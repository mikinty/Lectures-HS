The purpose of this worksheet is to familiarize you with the basic idea of algebra. Please know how to do everything here because it is very \textit{important} to your understanding of this topic.

\subsection{Variables}
When you think of Algebra, you probably think of something like this:

$$\sum_{i=1}^{34} 8x_{i}^{3}+\frac{i}{\sqrt{x_{i}^4}}+\prod_{k=1}^{34} (k)(k+1)(k+2)x_{k}^3 = x_{i,j}$$

Ok...so yes, Algebra uses symbols, but what do they mean? (by the way, as complicated as the expression looks above, we'll probably learn how to work with them in a few months! Just believe (and work hard)!)

In essence, Algebra is ``playing with the unknown''---we try to manipulate things we don't exactly know about. We call things that represent numbers \textbf{variables}. So if I had the expression
$$x+y=9,$$
then $x$ and $y$ are variables. Notice how even though we don't know the value of $x$ or $y$, we can still get some information from the expression. We know that the sum of $x$ and $y$ is 9. I know this seems really obvious, but it's \textit{very important} that we can write equations that give us information about variables that we don't know the value of.

To get you comfortable with working with variables, try the following exercises. They are essentially ``plug and chug'' questions---plug in and then evaluate---but they test your understanding and familiarity with how variables represent things.

\subsubsection{Problems}
Find the value of the following expressions. Set $x=3$.
\begin{multicols}{2}
\begin{enumerate}
\item $x+8$
\item $x^2$
\item $(x+7)^2$
\item $25\times(x+21)$
\item $\sqrt{x+13}$
\item $\frac{x+1}{x-1}$
\end{enumerate}
\end{multicols}

\subsection{What we actually use variables for}
So you might be wondering...wow...variables seem so \textit{useless}.
$$\text{No. They are the most useful thing ever.}$$
To demonstrate this fact, let's play a game!

\begin{itemize}
\item Enter the first three digits of your phone number (after the 703)
\item Multiply that number by 80
\item Add 1
\item Multiply all of this by 250
\item Add the last four digits of your phone number 
\item Add the last four digits of your phone number again
\item Subtract 250
\item Divide by 2
\item MAGIC! It's your phone number!!!
\end{itemize}

Ok, as amazing as this is (I know what you guys are thinking already...this is lame), it's actually really simple and is no product of magic.

If I were to write out the steps into Algebra, using \textbf{variables} to keep track of what's happening, then maybe we can find out what's behind this! 

I am setting the first three digits of the phone number as $a$ and the last four digits as $b$.

\begin{align*}
\text{Phone Number} &= \frac{(80a+1)250 + b+b-250}{2} \\
&= \frac{20000a+250 + 2b-250}{2} \\
&= \frac{20000a+ 2b}{2} \\
&= \boxed{10000a+b}.
\end{align*}

Notice how the result obviously gives you your phone number. Algebra is no trickery! 

So in conclusion, we use variables because

\begin{enumerate}
\item They are easy to use (wanna type your phone number over and over again?)
\item They help us see \textit{why} things happen
\item You can make them \textit{general}, that is, they apply to any number of cases. The phone number trick was an example of how our result told us that it would work for any phone number
\end{enumerate}

\subsubsection{Problems}
\begin{enumerate}
\item Try to prove the ``5'' squaring trick! To start you off, try to expand $(10a+5)^2$, where $a$ represents all of the digits before the units digit (which is 5).
\end{enumerate}

\subsection{Final Remarks}
Since this isn't a full-fledged lecture, I will end with some closing words. Since algebra doesn't represent anything special---I know expressions like $x^{3/2}-9$ look funky---the variables are just \textit{numbers}. So at the end of the day, we can derive many properties that we learned in grade school about numbers that apply to algebra, because \textit{Algebra is literally just working with numbers in another ``language''}.

That means we still have:
\begin{itemize}
\item Associative Rule: $(a+b)+c = a+(b+c)$, $(ab)c = a(bc)$
\item Commutative Rule: $a+b=b+a$, $ab=ba$
\item Distributive Property: $ a(b+c) = ab+ac $
\item Additive Identity: 0
\item Multiplicative Identity: 0
\end{itemize}

So I'm guessing most of you guys looked at the list and decided it was too easy and you didn't need to read it.
Wrong. Please understand and know everything on the list.

Because variables are just numbers, we can basically do anything we normally do with numbers. For example, if we wanted to simplify the expression: 

$$\frac{1}{a^2} + \frac{5c}{b} + (a-b)(c-5b)$$

We can just proceed with what we would normally do (notice we will have to do FOIL now...I will go over this someday, but basically since you can't just make binomial expressions simpler. You have to work with multiplying binomials step by step).

\begin{align*}
\frac{1}{a^2} + \frac{5c}{b} + (a-b)(c-5b) &= \frac{b}{a^2b}+\frac{5a^2c}{a^2b}+\frac{a^2b(ac-bc-5ab+5b^2)}{a^2b}\\
&= \frac{1}{a^2b}(b+5a^2c+a^3bc-a^2b^2c-5a^3b^2+5a^2b^3)
\end{align*}

Notice how all of the properties I listed above still work, and notice how for fractions, we still do what we normally do with numbers---we find the common denominator.

\subsubsection{Problems}
Simplify the following expressions (they might end up being ugly but try to simplify them)
\begin{enumerate}
\item $5x+5y+3z+3y+14x-3y+8z$
\item $(2838x^{38}+287y-87392z^{x+283-y^2})\times0$
\item $ab+7ac-8ab+14ac$
\item $x+x+x^{2}+x^{3}+5x-3x^2+7x^3$
\item $\frac{3xy}{6xz}$
\item $\frac{a}{b}+\frac{4ab}{b^2}$
\end{enumerate}